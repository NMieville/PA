\documentclass[a4paper, french, 10pt]{article} % Type du document

% compiler avec : pdflatex, bibtex, pdflatex, pdflatex


% +---------------------------------------------------------------+
% | Language
% +---------------------------------------------------------------+
\usepackage[T1]{fontenc}
\usepackage[utf8]{inputenc}
\usepackage[french]{babel}
\usepackage{graphicx}
\usepackage{multirow}
\usepackage{subfiles}
\usepackage{bytefield}
\usepackage{lastpage}
\usepackage[
backend=biber,
style=ieee,
sorting=none
]{biblatex}
\addbibresource{biblio.bib}

\usepackage[top=2.5cm, bottom=2.5cm, left=2.5cm, right=2.5cm, showframe=false]{geometry}


%% A PERSONNALISER %%%%%%%%%%%%%%%%%%%%%%%
\renewcommand{\author}{Nathan Miéville}
\newcommand{\prof}{Prof. Marizio Tognolini}
\renewcommand{\title}{Leaf Wetness Sensor}
%%%%%%%%%%%%%%%%%%%%%%%%%%%%%%%%%%%%%%%%%%

\usepackage{fancyhdr}
\setlength{\headheight}{27.06pt}
\pagestyle{fancy}
% Header:
\renewcommand{\headrulewidth}{1pt}
\fancyhead[L]{\title} 
\fancyhead[R]{} 
% Footer:
\pagenumbering{arabic}
\fancyfoot{} % clear all footer fields
\renewcommand{\footrulewidth}{1pt}
\fancyfoot[R]{Page \thepage~/~\pageref{LastPage}} 
\fancyfoot[L]{\author }

\begin{document}

\thispagestyle{empty}

\begin{figure}[ht]
	\begin{minipage}[c]{.49\textwidth}
		\includegraphics[width=.7\linewidth]{figures/logo_mse}
	\end{minipage}%
	\hfill
	\begin{minipage}[c]{.49\textwidth}
		\raggedleft
		\includegraphics[width=.65\linewidth]{figures/logo_hesso}
	\end{minipage}%
\end{figure}

\begin{raggedright}
\begin{small}
Master of Science HES-SO in Engineering\\
Av. de Provence 6\\
CH-1007 Lausanne\\
\end{small}
\end{raggedright}

\vfill

\begin{raggedleft}
\begin{huge}
\textbf{Master of Science HES-SO in Engineering}\\
Orientation : Electrical Engineering (EIE)\\
\end{huge}

\vfill

{\Huge \title}\\

\vfill

{\large Fait par}\\[10pt]
{\Huge \textbf{\author}}\\[20pt]
\large Sous la direction de\\
\prof\\
Dans le groupe de recherche Systèmes embarqués communicants de la HES-SO Valais
\vfill

\large Lausanne, HES-SO Master, 2022\\
\end{raggedleft}

\newpage
\thispagestyle{empty}
\
\newpage
\thispagestyle{empty}
Accepté par la HES-SO Master (Suisse, Lausanne) sur proposition de\\[10pt]
Professeur \prof, conseiller du projet d'approfondissement\\[30pt]
Lausanne, le \today\\[40pt]

\begin{tabular}{lll}
\prof & \hspace{4cm} & Philippe Barrade\\
Conseiller du PA & \hspace{4cm} & Responsable de la filière Electrical Engineering\\
\end{tabular}

\newpage
\thispagestyle{empty}
\section*{Remerciements}
...

\newpage
\thispagestyle{empty}
\tableofcontents

\newpage

\section{Introduction}
L'Agriculture doit évoluer pour faire face aux défis de notre époque. De nombreuses solutions sont en développement pour aider les agriculteurs à produire durablement, en respect avec la nature et en utilisant le moins de ressource possible tout en gardant une production suffisante pour atteindre la souveraineté alimentaire. L'agriculture de précision est une des solutions à toutes ces problématique. L'objectif est d'utiliser la technologie pour produire mieux. En observant et en mesurant le plus de variable possible nous somme capable, grâce à des modèles, à prévoir les besoins d'une plantation et ainsi mieux optimiser la production en consommant moins de ressource tel que l'eau les engrais ou les pesticides. 

Pour répondre à cette demande JDC Electronic s'est lancé le défit de proposer une solution complète, facilement déployable par les agriculteurs. Nous voulons proposer un assortiment de capteur centralisé sur une seule plateforme. Cette plateforme embarquera plusieurs modèle pour différent aspect de la production. L'interface sera facilement lisible pour un agriculteur et donnera des indications claire en temps réel sur les moyens d’optimisations. Un projet innosuisse à été lancé en 2021 en collaboration avec Agroscope et l'institut IICT de l'heig-vd . Il portait essentiellement sur la mesure et l'analyse du diamètre d'un fruit au cours de sa croissance en foncions de son arrosage. L'objectif est d'avoir un meilleur contrôle sur l'eau. Ce projet à permis le développement de écosystème des capteurs et de la plateforme et à abouti à des mesures en conditions réels dans des plantations de professionnels. Les étapes suivantes sont l’agrandissement de la gamme de capteur, la mise en production puis la commercialisation.

Une des voies que JDC Electronic aimerait développer est la prévention des maladies agricole. Les plantations sont les proie de nombreuse maladie qui peuvent être dévastatrice. Des modèles existent pour aider les agriculteurs à traiter au bon moment et éviter de devoir retraiter de nombreuse fois ou même de perdre leur production. L'un des modèles les plus utilisé \cite{doi:10.1094/PDIS-05-14-0529-FE}  se base sur la durée d'humectation. C'est à dire sur le temps que les gouttes créer par la rosée ou la pluie reste sur la feuille. Un capteur d'humectation permet d'obtenir cette grandeur.

L'objectif de ce travail sera d'effectuer une pré-étude afin de déterminer la faisabilité d'un tel capteur, de définir un cahier des charges et de fournir une preuve de concept. Nous commencerons par étudié ce qui se fait déjà avec un état de l'art. Les première piste  nous permettrons d'effectué une analyse fonctionnel afin de définir clairement nos besoin et nos contrainte et ainsi être sûre de ne rien oublier. Le cahier des charges sera défini. Dans une partie conception nous mettrons en pratique ce que nous a appris notre analyse afin d'obtenir une vue concrète de notre capteur. Nous pourrons mettre en place une simulation afin de définir les dernières variables. Et enfin, nous effectuerons des mesure sur une collection de configuration choisi avec la simulation afin de présenter une preuve de concept.


\section{Etat de l'art}

resistif:
plusieur constructeur:
Davis,Spectrum Caipos Lw
grille mesure electrique:
+ pas cher
- ne détecte pas les fine coute -> peinture
faux positif avec l'humidité

Metos: deux électrode et un tissus -> même problème

capacitif:

Meter (ancienement DECAGON) PHYTOS31 
un des seul du marché beaucoup de revendeur

une expérimentation de: Instrumentation, Sensor and Interfaces Group, Universitat Politècnica de Catalunya, BarcelonaTech, Spain 


\graphicspath{ {./figuresAnalysis} }
\section{Analyse fonctionnelle}
\subsection{Besoins}
Afin d'être sûre de commencer dans la bonne direction nous devons définir clairement les besoins auxquels notre capteur devra répondre. Comme décrit précédemment, le besoin principal est la prévention du développement de maladie. Il existe des modèles empiriques qui se basent sur le temps d'humidité sur la feuille. Cette variable est difficile à déterminer par les données météo classique (humidité, température, vents etc). Le recours à un capteur d'humectation est utile dans ce cas là.
Un autre besoin est pour l'aide aux traitement. La plupart des traitements chimiques d'une plantation doivent être effectués lorsque les feuilles sont sèches. Un capteur d'humectation doit permettre de fournir cette information facilement.  

\begin{figure}[!ht]
 \centering
 \includegraphics{DiagrammeCorne.drawio.pdf}
 \caption{Le besoin principale exprimé sous la forme d'un diagramme bête à cornes.}
\end{figure}

\subsection{Fonction principale et contrainte }

Pour répondre aux besoins, la fonction principale sera de mesurer l'humectation des feuilles. Cette fonction s'accompagne de plusieurs contraintes apportées par l’environnement dans lequel s'inscrit le capteur. Pour être sûre de n'en oublier aucune, nous nous aidons d'un diagramme pieuvre. 

\begin{description}
 \item[FC1] Il est développé dans le cadre du projet JDC Smart Farming. Il devra être compatible avec le système déjà créé. 
 \item[FC2] Il sera déployé dans des plantations avec une batterie comme source d'énergie. La consommation doit être contrôlée.   
 \item[FC3] En extérieur la météo peut faire varier l’environnement du capteur. Il devra être robuste à ces changements pour qu'il n’influence pas les mesures.
 \item[FC4] Les intempéries que subira le capteur ne doivent pas l’endommager ou compromettre les mesures.
 \item[FC5] Dans les plantations, il y a régulièrement des tracteurs et des machines qui passent entre les plantes. Le capteur ne doit pas gêner ou être gêner par ces passages. Sa taille doit être contrôlée.
 \item[FC6] L'installation et la maintenance pourra être faite par des agriculteur sans formation technique. Le capteur doit être simple d'installation et de maintenance.
 \item[FC7] Le capteur est développé pour être commercialisé. Il doit répondre aux norme et être certifié.
\end{description}

\newpage

\begin{figure}[!ht]
 \centering
 \includegraphics{DiagrammePieuvre.drawio.pdf}
 \caption{Fonctions principales et contraintes sous la forme d'un diagramme pieuvre}
\end{figure}

Après avoir énoncé la fonction principale et toutes les contraintes nous pouvons définir des critères pour chacune d'elles ainsi que des nivaux qui répondent à ces critères. Cela nous permet de construire le cahier des charges auquel nous nous référons pour la conception. 

\paragraph{FP1}
La mesure d'humectation des feuilles se fera au travers d'une mesure de l'humidité relative d'une surface. Elle se donne en pourcentage tel que 0\% correspond à un feuille totalement sèche et 100\% la feuille est entièrement recouverte d'eau. Puisque nous ne connaissons pas encore les performances de notre capteur nous utiliserons la résolution maximale qu'autorise la structure de registre de JDC pour ce type de capteur. La valeur sera stockée sur 1 octet non signé avec une résolution de 0.5\%. La précision est calqué sur la résolution et donne une valeur de 0.25\%. Ces valeur sont très optimistes et nous nous resservons le droit de les changer après une évaluation des performance du capteur plus tard dans le développement. 

\paragraph{FC1} \label{fc1}
L’environnement Smart Farming JDC se compose d'un émetteur LoRa auquels sont reliés plusieurs capteurs au travers d'un bus I2C. Pour que notre capteur soit compatible, il doit impérativement répondre à plusieurs critères. L'interface de sortie doit être évidement un I2C. La structure des registres accessibles est normalisé pour que l’émetteur puisse lire correctement les valeurs afin de les transmettre. Un temps maximal de mesure est défini. Il représente le temps entre le démarrage du capteur jusqu'à que les valeurs de la mesure soit prête. Le capteur devra être câblé sur le connecteur commun à tous les capteurs JDC. Pour finir, l'alimentation fourni par l'émetteur est de 3.3V le capteur devra fonctionner à cette tension.

\paragraph{FC2}
La source d'alimentation du capteur sera une batterie situé aux niveau de l'émetteur. Tous les capteurs d'un même émetteur partage donc la même source. Les capteurs ne sont pas alimentés entre deux mesures. Nous n'avons pas besoin de nous préoccuper de la consommations au repos. En marche, le courant, que nous prendrons comme critère, ne doit pas dépasser 1mA. Ce chiffre avait été calculé par rapport au nombre maximal de capteur, la capacité de la batterie et l'autonomie souhaité.  

\paragraph{FC3}
Le facteur métrologique qui pourra le plus fausser nos mesures est l'humidité de l'air. Si l'air est chargée en eau, sa constante diélectrique changera et pourra impliquer un augmentation de la capacité alors que la surface est complètement sèche. Pour que ce phénomène n’influence en aucun cas nos mesures, le delta de capacité doit être inférieur à la précision. Cette variation devra être contrôlé et mesurée car elle pourrait dégrader la précision du capteur.   

\paragraph{FC4}
L'utilisation extérieur du capteur nécessite qu'il soit étanche aux intempéries. Avec la norme IP65 comme objectif, le capteur sera suffisamment protéger des plus grosses pluies ainsi que des traitements pulvérisés sur les cultures.    

\paragraph{FC5}
Pour s’intégrer aux mieux dans les plantations le capteur ne devra pas être trop volumineux. Nous prendrons arbitrairement une envergure maximum de 20 cm. Cela correspond à une moyenne des capteur existant sur le marché.

\paragraph{FC6}
La facilité d'installation a déjà été pensée et est garantie par la contrainte \textbf{FC1}.

\paragraph{FC7}
Pour être proposé sur le marché le capteur doit être certifié. Il doit possédé le CE pour être distribué en Europe et une certification de compatibilité électromagnétique pour garantir que le capteur respecte les normes en vigueur.  

\bigskip
Le cahier des charge résume tous les critères et leur niveau pour chaque contrainte.

\begin{table}[!h]
\begin{center}
\resizebox{\columnwidth}{!}{%
\begin{tabular}{|l|l|p{5cm}|l|}
\hline
 & \textbf{Fonctions} & \textbf{Critères} & \textbf{Niveaux}\\
\hline
\textbf{FP1} &Mesurer l'humectation des feuilles & Mesure d'humidité relative d'une surface & RH de 0\% à 100\%  résolution de 0.5\%  précision +- 0.25\%\\
\hline
\multirow{5}{*}{\textbf{FC1}} & S'intégrer dans l'environnement Smart Farming JDC & Interface  de sortie I2C & Baud rate 100KHz. Adresse configurable\\\cline{3-4}
 &  & Structure de registre normalisé, & (voir doc JDC)\\\cline{3-4}
 &  & Démarrage de la mesure et acquisition après un temps. & 50ms pour la capture de la mesure\\\cline{3-4}
 &  & Connecteur JDC & Sortie 4 fil avec  VCC,GND,SDA,SCL\\\cline{3-4}
 &  & Alimentation normalisé & Tesion 3.3V\\\cline{3-4}
\hline
\textbf{FC2} & Consommer peu d'énergie & Courant maximum établit en fonctionnement & 1 mA\\
\hline
\textbf{FC3} & Eviter les faux positifs du à la météorologie & L'humidité de l'air ne doit pas influencer la mesure & L'incidence de RH de l'air < précision (0.25\%)\\
\hline
\textbf{FC4} & Résister aux milieux extérieurs & Le capteur est protégé des intempéries et supporte une utilisation extérieur & Etanche IP65\\
\hline
\textbf{FC5} & S'intègrer dans les plantations & La taille du capteur ne doit pas gêner l'exploitation des plantations & Envergure maximum de 20cm\\
\hline
\textbf{FC6} & Etre facile d'installation & Le capteur doit pouvoir être installer par des agriculteurs sans formation technique & Système d'attache et un seul connecteur à brancher\\
\hline
\textbf{FC7} & Etre Certifié & Le capteur doit être certifié pour être proposé sur le marché & Certification EMC,CE\\
\hline
\end{tabular}
}
\end{center}
\caption{Cahier des charges}
\end{table}

\newpage

\subsection{Architecture du système}

En nous basant sur le cahier des charges, nous pouvons commencer à concevoir l'architecture de notre capteur. Pour nous aider dans cette tâche nous utiliserons un diagramme FAST. Il nous permettra méthodiquement de faire la liste de tous les éléments nécessaires au bon fonctionnement du capteur.

\begin{figure}[!ht]
 \centering
 \includegraphics[width=16cm]{DiagrammeFAST.pdf}
 \caption{Diagramme FAST}
\end{figure}

Plusieurs choix techniques ont été fait pendant cette phase d'architecture. Le MAX32660 a été choisi comme micro-contrôleur. JDC l'utilise déjà dans ses capteurs. Plusieurs librairies ont déjà été créés pour intégrer le capteur dans l’environnement JDC ce qui facilitera le développement. Le micro-contrôleur a tous les périphériques que nous avons besoin et est prévu pour la faible consommation. Il ira très bien dans notre application.


l'AD7150 a été choisi parmi les différents convertisseurs capacitifs présentés pendant l'état de l'art (datasheet: Annexe \ref{dataad}). La sélection s'est faite sur la consommation des circuits intégrés. l'AD7745 et le FDC1004 ont une consommation en travail de  \SI{900}{\micro\ampere}. C'est beaucoup trop si on prends en compte le fait qu'un micro-contrôleur s'ajoute à la consommation. Nous serions au-dessus des \SI{1}{\milli\ampere}. l'AD7150 consomme \SI{100}{\micro\ampere} ce qui nous laisse plus de marge. Une carte de développement est disponible à l'achat afin de réaliser nos premières mesures. A ce stade il est trop difficile d'estimer la capacité que nous aurons à mesurer. Nous prendrons les bornes de ce capteur comme référence pour la création du dipôle. Si nous observons que nous sommes complètement en dehors, nous réévaluerons le choix du convertisseur.

L'alimentation provient du transmetteur et elle est transporté à travers un câble. Pour une mesure correcte, l'alimentation doit être le plus propre possible. Un nettoyage de la tension d'alimentation devra être fait. Nous n'aurons pas l’occasion d'étudier ce circuit dans ce travail car nous utiliserons pour les mesures des cartes de développement qui embarquent leur propre alimentation. Néanmoins, cette partie ne doit pas être négligé pendant la suite du développement.

\newpage
Nous pouvons dès à présent mettre tous nos choix bout à bout pour établir un Schéma block sur lequel nous nous baserons pour la conception. Une recherche dans la datasheet du circuit de mesure de la capacité nous apprend les bornes exacte ainsi que la résolution numérique de la valeur mesurée. Nous pouvons alors tracer la chaîne complète de mesure de notre capteur.

\begin{figure}[!ht]
 \centering
 \includegraphics[width=16cm]{SchemaBlock.pdf}
 \caption{Schéma Block}
\end{figure}


\graphicspath{ {./figuresConception} }
\section{Conception}

\subsection{Circuit}
L'objectif de ce travail est d'analyser et de valider le principe physique de notre capteur. Nous nous concentrerons principalement sur le dipôle. Nous établissons un schéma électrique du capteur pour nous aider lors du câblage des cartes de développement. Ce schéma ne comporte pas la gestion de l'alimentation car c'est la chaîne de mesure qui nous intéresse.

\begin{figure}[!ht]
 \centering
 \includegraphics[width=14cm]{schemaelec.png}
 \caption{Schéma électrique}
\end{figure}

Sur ce schéma C$_4$ représente le dipôle. Il est connecté sur le premier port du convertisseur. Un pôle est connecter sur la pin d'excitation et l'autre sur l'entrée analogique. La conversion est faite en excitant d'un côté et en mesurant la charge de l'autre. La valeur est stockée dans un registre prête à être lu par l'I2C. Un filtre conseillé par la datasheet est câblé sur l'alimentation de l'AD7150. Le micro-contrôleur est connecté aux convertisseurs par les deux pin de l'I2C, clock et donnée. On utilise le deuxième I2C pour le convertisseur et le premier pour l'interface extérieur. Les pull-up obligatoires sont connecté sur le deuxième I2C mais pas sur le premier car elle se trouve déjà aux niveau de l'émetteur. Les deux sorties numériques sont aussi connecté au micro-contrôleur. Nous ne les utiliserons pas tout de suite mais nous nous en laissons la possibilité. Le connecteur P1 est un tag pour flasher le micro. Il est relié aux pin nécessaire pour sa tâche et à l'I2C d'interface à des fin de debug. Et finalement le connecteur J1 permet de sortir l'interface I2C et apporte l'alimentation. Des diodes protègent le circuit contre les surtensions.  

\newpage
\subsection{Dipôle}
Le dipôle est la partie de notre capteur qui transforme une variable physique en un paramètre électrique. Dans notre cas il transforme une quantité d'eau sur une surface en une variation de capacité. Pour comprendre comment cela fonctionne il faut comprendre ce qu'est une capacité électrique. $C=\frac{Q}{U}$ défini la capacité (C) comme la charge (Q) stocké par rapport à une tension donné  (U) . La capacité d'un dipôle est influencé par trois paramètres. La surface du conducteur l'augmentera. La distance entre deux pôles la diminuera. Et enfin, une plus grande permittivité relative du milieux augmentera la capacité. L'eau fera varier ce dernier paramètre ce qui nous laisse les deux autres à définir.

Le dipôle sera d'abord simulé puis mesuré. La simulation se fera avec le logiciel Flux d'Altair. Pour simplifier la simulation nous l’exécuterons en 2 dimension. Ce premier dipôle devra être conçu pour qu'une coupe 2d puisse être utilisé dans la simulation. Nous choisirons un peigne droit en longueur. 

\begin{figure}[!ht]
 \centering
 \includegraphics[width=14cm]{dipole-top.pdf}
 \caption{Plan mécanique du dipôle}
 \label{plan}
\end{figure}

Nous avons défini arbitrairement une longueur de 100mm et une largeur de 60mm. Cette taille nous donne un bon compromis entre l'espace disponible sur la carte et le prix de production. 

\begin{figure}[!ht]
 \centering
 \includegraphics[width=14cm]{dipole-A-A.pdf}
 \caption{Plan mécanique de coupe du dipôle}
 \label{plancoupe}
\end{figure}

Les différentes épaisseurs des couches, du PCB, du cuivre et du masque sont récupérés à partir des dimension de carte standard. C'est cette coupe que nous simulerons dans le logiciel. Cependant nous ajouterons une couche de résine supplémentaire de 0.5 mm d'épaisseur. Elle permettra sur le capteur final de protéger le circuit et d'offrir à l'eau une surface proche de celle d'une feuille en terme d'étanchéité et de rugosité. Cette couche ne sera pas présente pour nos premières mesures. Mais doit être prise en compte pendant la simulation. Nous évaluerons aussi l'utilisation d'un plan conducteur pour concentrer le champs en direction de la surface mesurée et éviter que le capteur soit influencé par ce qui se passe en dessous. Nous prendrons la décision de garder ou non cette couche après l'avoir simulé et étudié.

Le nombre, l'épaisseur, et l'écart des pistes sont des valeurs que nous laissons libre pour la simulation. La coupe ne représente que la partie centrale du PCB. Pour une question de simplicité nous négligerons les extrémités pour les simulations.



\graphicspath{ {./figuresSimulation} }
\section{Simulation}
\subsection{Création du modèle}
Pour simuler notre dipôle nous utiliserons Flux2D d'Altair. La simulation se fait par élément fini. Cela consiste à dessiner notre géométrie pour la diviser en plusieurs points grâce à un maillage. Le logiciel résout en chaque point une équation différentielle dérivée des équations de Maxwell pour une application donnée. Pour mesurer une capacité nous simulerons dans le domaine de l'électrostatique. Nous nous intéresserons aux charges et aux champs électrique.

\subsubsection{Géométrie}
Flux nous permet de faire du dessin paramétrique. Nous pourrons ainsi, pendant la simulation, faire varier nos paramètres géométriques. Nous créons des paramètres pour toutes nos grandeurs: épaisseur et largeur. La profondeur est entrée à la création du modèle. Le logiciel intègre le fait que nous dessinons une coupe et utilisera la profondeur  pour les calcul qui concerne l'ensemble du système comme par exemple l'énergie total. Nous utilisons les valeurs défini pendant la conception, figure \ref{plan} et \ref{plancoupe}. Dans l’intérêt de la simulation nous ajoutons une couche d'eau. Nous ne pouvons pas simuler une goutte en deux dimensions. Nous étudierons un cas sans eau notre 0\% et un cas avec une couche continue notre 100\%. 

Le nombre de piste ne peut pas être paramétrisé car il crée des surfaces supplémentaires, ce qui n'est pas géré en simulation par Flux2D. Nous créerons trois projets pour trois nombre de piste: 20,10,5. Puisque la largeur du circuit et le nombre de pistes est fixe la somme distance et largeur de piste sera fixe aussi. Elle vaudra $S_{piste} = \frac{L_{pcb}}{N_{piste}}$. Si nous changeons la largeur des pistes nous devrons changer la distance. Nous utilisons alors un seul paramètre qui fera varier les deux. Il représente le rapport entre largeur et distance $F_{dist,larg} = \frac{L_{piste}}{D_{piste}} $. Nous pouvons définir largeur et distance en fonction de nos deux nouveaux paramètres: $D_{piste} = \frac{S_{piste}}{F_{dist,larg}+1}$, $L_{piste} = S_{piste} - \frac{S_{piste}}{F_{dist,larg}+1}$. C'est ces deux expressions que nous insérons dans le logiciel. 
Une fois tous les paramètres rentrés, nous construisons avec ceux-ci toute notre géométrie. On crée d'abord les points, puis on les relie pour former les lignes qui formerons nos surfaces. 

\begin{table}[!ht]
\begin{center}
\begin{tabular}{|l|l|}
\hline
Paramètre & valeur [mm]\\
\hline
EP\_CIBLE & 2.7\\
\hline
EP\_CUIVRE & 0.018\\
\hline
EP\_MASQUE & 0.035\\
\hline
EP\_PCB & 1.55\\
\hline
EP\_RESINE & 0.5\\
\hline
LARG\_PCB & 60\\
\hline
NBR\_PISTE & 20,10.5\\
\hline
FACTOR\_DIST\_LARG\_PISTE & variable\\
\hline
SUM\_DIST\_LARG\_PISTE & LARG\_PCB/NBR\_PISTE\\
\hline
LARG\_PISTE & {\tiny SUM\_DIST\_LARG\_PISTE-(SUM\_DIST\_LARG\_PISTE/(FACTOR\_DIST\_LARG\_PISTE+1))}\\
\hline
DIST\_PISTE & {\tiny SUM\_DIST\_LARG\_PISTE/(FACTOR\_DIST\_LARG\_PISTE+1)}\\
\hline
\end{tabular}
\caption{Liste des paramètre}
\end{center}
\end{table}

\begin{figure}[!ht]
 \centering
 \includegraphics[width=14cm]{c20geotout.png}
 \caption{Géométrie du modèle entier}
\end{figure}

\begin{figure}[!ht]
 \centering
 \includegraphics[width=14cm]{c20geozoom.png}
 \caption{Géométrie du modèle, zoom sur les pistes}
\end{figure}

\begin{description}
 \item[Bleu Foncé:] couche cible (eau)
 \item[Noir:] Résine
 \item[Vert:] Masque de soudure
 \item[Rouge:] Piste pôle 1
 \item[Bleu claire:] Piste pôle 2
 \item[Brun:] PCB
 \item[Turquoise:] Plan cuivre
\end{description}

Une dernière surface devra être créée pour représenter l'infini. Flux2d fonctionne en appliquant des conditions aux limite sur une sphère défini par l'utilisateur entourant notre modèle. Ces conditions permettent de simuler un environnement infini dans un zone fini. 

\subsubsection{Physique}
Une fois la géométrie terminée nous pouvons ajouter des contraintes physiques à notre modèle. Cela consiste à assigner des matériaux à nos surfaces. La résine, le masque et le pcb sont des isolants avec une certaine constante diélectrique. Le cuivre sera considéré comme un conducteur parfait. Une tension sera appliquée sur les deux pôles. Nous fixerons une tension de 3.3V et une autre de 1.65V. Cela nous donnera une différence de 1.65V. Nous l’enregistrerons dans un paramètre physique qui pourra être utilisé plus tard dans les calculs. Le plan de cuivre restera flottant. L'eau sera aussi considéré comme un isolant avec une constante diélectrique. La cible, le plan et la résine pourra être remplacé par de l'air au besoin de la simulation.


\begin{table}[!ht]
 \begin{center}
\begin{tabular}{|l|l|l|l|}
\hline
Surface & Matériau & $\mu_r$ & Potentiel\\
\hline
Cible & eau & 80 & -\\
\hline
Résine & résine & 4 & -\\
\hline
Masque & résine & 4 & -\\
\hline
Piste P1 & conducteur parfait & - & 3.3 V\\
\hline
Piste P2 & conducteur parfait & - & 1.65 V\\
\hline
PCB & FR4 & 4.4 & -\\
\hline
Plan & conducteur parfait & - & flotant\\
\hline
\end{tabular}
\caption{Caractéristiques physique}
\end{center}
\end{table}



\subsubsection{Maillage}
Le maillage est une partie importante de la création d'un modèle du simulation par élément fini. Un compromis doit être trouvé entre le temps de calcul et la résolution de la simulation. Quelques règles assez simple permettent d'obtenir des résultats convenables. Plus le champs est intense à un endroit plus le maillage doit être fin. Il faut aussi être fin aux intersections entre deux matières. Les grandes surfaces peuvent être plus relâchées aux centre. En suivant ces recommandations, nous obtenons un maillage satisfaisant.


\begin{figure}[!ht]
 \centering
 \includegraphics[width=14cm]{C20maillage.png}
 \caption{Maillage du modèle}
\end{figure}


\subsection{Exploitation}

\subsubsection{20 pistes}
Nous pouvons à présent démarrer les simulations. Nous commencerons avec 20 pistes et un rapport entre largeur et distance de 2 (largeur des pistes = 2mm, distance entre piste = 1mm). La cible ainsi que le plan seront en air. Une fois la simulation terminée, nous pouvons afficher le champs magnétique ainsi que les lignes du potentiel électrique pour analyser le résultat et vérifier que la simulation s'est bien passé.

\begin{figure}[!ht]
 \centering
 \includegraphics[width=14cm]{simulationChampElectrique.png}
 \caption{Champs électrique 20 pistes}
\end{figure}

\newpage
\begin{figure}[!ht]
 \centering
 \includegraphics[width=14cm]{C20air.png}
 \caption{Lignes de potentiel électrique 20 pistes}
\end{figure}

On remarque premièrement sur ces graphiques que le maillage est correct. Les lignes sont bien continue même très proche des conducteurs. Les coupures des niveaux du champs magnétique sont dues à la transition d'un milieu à un autre et sont tout à fait normal. La deuxième information que nous fournit ces figures sont le fait que le champs est bien présent dans l'espace de la cible. Cela veut dire qu'un changement de caractéristique à ce niveau aura une influence sur le champs et donc la capacité. C'est positif pour la viabilité de notre capteur.

Nous pouvons calculer dans cette simulation la capacité de notre système. Nous utilisons pour cela la définition de la capacité qui dépend de l'énergie $C = \frac{2W}{U^2}$. Flux2d intègre tout les points pour calculer l'énergie électrique de tous le modèle en prenant en compte la profondeur spécifiée ultérieurement. Nous utilisons la tension que nous avions mis en paramètre pour attribuer l’expression de la capacité dans un paramètre qu'on appellera C\_total. Ce paramètre sera disponible pour nos prochaine simulation afin d'obtenir directement la valeur et tracer des graphiques. Pour cette première simulation C vaut \SI{78.98}{\pico\farad}. Nous somme au dessus de l'offset maximal de notre convertisseur qui est de 10pF. Nous somme cependant dans le bonne ordre de grandeurs et il ne sera pas difficile de diminuer la capacité. 

Nous pouvons jouer avec le rapport distance et largeur de piste afin d'observer comment la capacité varie. Nous créons pour cela un scénario de simulation pour faire varier notre paramètre de 0.5 à 2 sur 10 pas. On remarque bien une diminution de la capacité mais ce n'est pas suffisant. Nous observerons plus tard comment elle diminue encore en réduisant le nombre de piste.

\newpage

\begin{figure}[!ht]
 \centering
 \includegraphics[width=10cm]{C20airGraph.png}
 \caption{Capacité en fonction du rapport distance/largeur}
\end{figure}

Nous allons à présent changer la cible en eau afin d'observer les changements produits.


\begin{figure}[!ht]
 \centering
 \includegraphics[width=14cm]{C20eauligne.png}
 \caption{Lignes de potentiel électrique 20 pistes avec eau}
 \label{c20eaul}
\end{figure}

\begin{figure}[!ht]
 \centering
 \includegraphics[width=7cm]{C20eauGraph1.png}
 \includegraphics[width=7cm]{C20eauGraph2.png}
 \caption{Comparaison des capacité avec et sans eau pour 20 pistes}
\end{figure}

L'eau modifie bien le champs électrique. On vois sur la figure \ref{c20eaul} que le champs est comprimé. L'effet de l'eau sur la capacité est tout aussi visible. La capacité augmente de près de la moitié de sa valeur. Un fait intéressent est que la différence évolue plus vite que l'offset de capacité. Cela veut dire que si on diminue trop le rapport pour faire diminuer la capacité d'offset on risque de réduire beaucoup trop la sensibilité, bien que nous somme toujours au dessus du seuil de notre convertisseur.

\subsubsection{5 et 10 pistes}
Nous effectuons une simulation avec 5 et 10 piste sur la carte. Nous utilisons toujours le même scénario avec et sans eau.


\begin{figure}[!ht]
 \centering
 \includegraphics[width=7cm]{C10Graph1.png}
 \includegraphics[width=7cm]{C10Graph2.png}
 \caption{Comparaison des capacité avec et sans eau pour 10 pistes}
\end{figure}

Cette nouvelle configuration est meilleur que la première. la capacité d'offset a été divisé par deux tout en gardant un bon delta. Nous somme à présent plus proche des caractéristiques du convertisseur.

\begin{figure}[!ht]
 \centering
 \includegraphics[width=7cm]{C5Graph1.png}
 \includegraphics[width=7cm]{C5Graph2.png}
 \caption{Comparaison des capacité avec et sans eau pour 5 pistes}
\end{figure}
\newpage

Grâce à 5 piste nous atteignons enfin la capacité d'offset cherchée. La différence de capacité a diminué mais est toujours très élevée si on la compare à l'offset. Ces résultats sont à bien considérer car le modèle ne nous permet pas de simuler de fine goutte entre les lignes. Il est donc difficile de prévoir si une carte avec moins de ligne peut capter toutes les gouttes. Seules les mesures permettrons de nous le dire. 

\subsubsection{Plan de cuivre}

L'utilisation d'un plan de cuivre a été pensé pour concentrer la mesure à la surface haute de la carte. Nous appliqueront donc un conducteur parfait à la couche sous le pcb. Nous allons dans un premier temps simuler l'effet d'un plan seulement sur le modèle à 5 lignes pour comprendre son effet. Nous lancerons le même scénario avec et sans eau pour que nous puissions comparer.

\begin{figure}[!ht]
 \centering
 \includegraphics[width=10cm]{C5airMasseligne.png}
 \caption{Lignes de potentiel électrique, avec plan, avec eau pour 5 pistes}
 \label{c5planlair}
\end{figure}

\begin{figure}[!ht]
 \centering
 \includegraphics[width=10cm]{C5eauMasseligne.png}
 \caption{Lignes de potentiel électrique, avec plan, sans eau pour 5 pistes}
 \label{c5planleau}
\end{figure}

\newpage

\begin{figure}[!ht]
 \centering
 \includegraphics[width=7cm]{C5masseGraph1.png}
 \includegraphics[width=7cm]{C5masseGraph2.png}
 \caption{Comparaison des capacité, avec plan, avec et sans eau pour 5 pistes}
\end{figure}

L'ajout d'un plan augmente l'offset et diminue le delta ce qui n'est pas très intéressant pour nous. Cependant on voit sur la figure \ref{c5planlair} et \ref{c5planleau} que les lignes de potentiel ne sortent plus par dessous. On en déduit grâce au graphique que, comme prévu, tout ce qui se passe en dessous n’influencera plus notre capteur. Nous garderons cette configuration pour nos mesures. Afin d'obtenir des valeur de comparaison pour les mesures nous effectuons une dernière simulation avec le plan de cuivre pour 10 et 20 pistes. 

\begin{figure}[!ht]
 \centering
 \includegraphics[width=7cm]{C1020masseGraph1.png}
 \includegraphics[width=7cm]{C1020masseGraph2.png}
 \caption{Capacité, avec plan, avec et sans eau pour 10 et 20 pistes}
\end{figure}

Encore une fois, les performances avec 10 pistes sont meilleures qu'avec 20 piste. Avec le plan de cuivre, la configuration de 10 piste a toujours un plus faible offset et une plus grande sensibilité. Pour cette raison, nous n'utiliserons pas de 20 pistes pour la suite.


\section{Bibliographie}

\printbibliography

\end{document}
