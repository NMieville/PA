\section{Conclusion}

L'agriculture est phase de changer grâce au développement de nouvelle technologie. JDC électronique voudrait apporter ses solution pour aider les agriculteur à produire mieux avec moins de ressource. La prévention des maladies fait partie des domaine dans lequel des optimisations sont possible. Pour utilisé les modèle existant et conseillé sur le meilleur moment pour traiter une mesure de l'humidité du feuillage est nécessaire. C'est le rôle du capteur que nous avons étudié dans ce travail.

Il existe sur le marché principalement deux type de ce capteur. Résistif, le plus rependue, souffre de plusieurs problème de sensibilité à l'humidité de l'aire due à l'utilisation de couche absorbante. Les capteur capacitif sont beaucoup moins répandu. Un seul constructeur en commercialise. Ils sont moins bon marché et plus complexe à développer. C'est sur ce dernier type que notre choix c'est porté. Nous avons  utilisé le fait que le principe soit semblable au capteur tactile pour utilisé un circuit intégré convertissant une capacité en valeur numérique. Ce qui nous soulage de la partie la plus délicate de la mesure.

Nous avons défini le besoin auquel doit répondre le capteur ainsi que ses contrainte. Il doit être compatible avec l’environnement de capteur déjà développer par JDC et fonctionner en extérieur. Nous avons défini le schéma bloque en utilisant le max32660, un microcontrôleur déjà utilisé par JDC, et l'AD7150 comme convertisseur.

La partie principale que nous avons simulé est le dipôle dont nous mesurons la capacité. Un premier dessin mécanique d'un peigne droit sur la longueur à été établi. Les piste sont plus ou moins espacé en fonction de l'étude. Dans nos configurations nous changeons aussi le nombre de piste entre 5 10 et 20. Nous avons utilisé une coupe 2d pour les simulations.

Nous avons utilisé le logiciel de simulation a élément fini Flux d'Altaire. Une simulation en 2 dimension nous permet de commencer sur un modèle simple pour le complexifié si besoin plus tard. Le logiciel nous a permis de définir des paramètres géométrique pour étudier plusieurs configurations.Une attention particulière à été porter sur le maillage pour nous assuré une simulation correcte.Nous avons pu mesuré la capacité total entre les deux pole pour chaque scénario. L'étude du champs électrique nous a montré qu'une goutte d'eau influencerai celui-ci. Après avoir récolté la capacité de toute nos configurations nous observons une bonne sensibilité à l'eau mais la capacité est trop élever pour notre convertisseur. Nous varions entre 10pF et 60pF.

Nous avons sélectionné quelques configurations pour les produire en PCB. Pour notre première mesure nous insérons nos carte comme capacité dans un filtre RC. Nous avons due prendre en compte l’influence de la sonde et l'oscilloscope. Les capacité sont plutôt proche de celle mesuré nous avons entre 10\% et 20\% d'écart. Il existe des différence géométrique et électrique entre la simulation et la mesure qui explique les écart. Ces résultat nous donne une certaine confiance dans l'utilisation de la simulation. Pour de future simulation nous saurons à quoi nous attendre en réalité. Une carte de développement du convertisseur choisi est ensuite utilisé. Il en ressort que la mesure du convertisseur est trop influencer par la capacité parasite à la masse. Une mesure absolue donne trop d'écart avec les résultat du filtre RC et la simulation. Nous avons pu cependant l'utilisé en relatif afin de tester l'effet de goutte d'eau. Le capteur les détectent différemment suivant leur position et leur taille. Un travail d'optimisations sera nécessaire sur le peigne.

Ce travail nous a permis d'effectuer une première analyse sur ce capteur. Il nous a apporté une compréhension des phénomène en jeux ainsi qu'une certaine confiance dans la simulation et les technique de mesures. Ce sera un très bon départ pour la suite du développement du capteur.

Afin d'aboutir à un produit fini une simulation plus pousser sera nécessaire pour optimisé le peigne et la détection de petite goutte. Des test dans le plus de condition possible devrons être effectué afin de tester l’influence de l'humidité de l'aire sur la mesure. Une réévaluation du choix du convertisseur devra être faites. Le circuit complet sera développer en faisant attention à la gestion de l'alimentation. Le programme du microcontrôleur pour l'acquisition et l'interface devra être écris. Et enfin des test finaux devrons être effectuer afin de préparé le capteur à une certification pour une future commercialisation. Beaucoup de travail est encore nécessaire pour mener à terme ce projet mais c'est pré-étude à été très encourageante pour la suite. 
