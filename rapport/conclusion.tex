\section{Conclusion}

L'agriculture est phase de changer grâce au développement de nouvelle technologie. JDC électronique voudrait apporter ses solutions pour aider les agriculteur à produire mieux avec moins de ressource. La prévention des maladies fait partie des domaines dans lequel des optimisations sont possibles. Pour utiliser les modèles existant et conseiller sur le meilleur moment pour traiter une mesure de l'humidité du feuillage est nécessaire. C'est le rôle du capteur que nous avons étudié dans ce travail dont l'objectif était de déterminer la faisabilité d'un tel capteur et la définition d'un cahier des charge.

Il existe sur le marché principalement deux types de ce capteur. Résistif, le plus rependue, souffre de plusieurs problème de sensibilité à l'humidité de l'air due à l'utilisation de couche absorbante. Les capteurs capacitif sont beaucoup moins répandu. Un seul constructeur en commercialise. Ils sont moins bon marché et plus complexe à développer. C'est sur ce dernier type que notre choix c'est porté. Nous avons  utilisé le fait que le principe soit semblable au capteur tactile pour utilisé un circuit intégré convertissant une capacité en valeur numérique. Ce qui nous soulage de la partie la plus délicate de la mesure.

Nous avons défini le besoin auquel doit répondre le capteur ainsi que ses contraintes. Il doit être compatible avec l’environnement de capteur déjà développer par JDC et fonctionner en extérieur. Nous avons défini le schéma bloque en utilisant le max32660, un microcontrôleur déjà utilisé par JDC, et l'AD7150 comme convertisseur.

La partie principale que nous avons simulé est le dipôle dont nous mesurons la capacité. Un premier dessin mécanique d'un peigne droit sur la longueur a été établi. Les pistes sont plus ou moins espacées en fonction de l'étude. Dans nos configurations nous changeons aussi le nombre de piste entre 5 10 et 20. Nous avons utilisé une coupe 2d pour les simulations.

Nous avons utilisé le logiciel de simulation à élément fini Flux d'Altaire. Une simulation en 2 dimensions nous permet de commencer sur un modèle simple pour le complexifié si besoin plus tard. Le logiciel nous a permis de définir des paramètres géométrique pour étudier plusieurs configurations. Une attention particulière a été porter sur le maillage pour nous assurer une simulation correcte. Nous avons pu mesurer la capacité totale entre les deux pôles pour chaque scénario. L'étude du champs électrique nous a montré qu'une goutte d'eau influencerait celui-ci. Après avoir récolté la capacité de toute nos configurations, nous observons une bonne sensibilité à l'eau mais la capacité est trop élever pour notre convertisseur. Nous varions entre 10pF et 60pF.

Nous avons sélectionné quelques configurations pour les produire en PCB. Pour notre première mesure, nous insérons nos carte comme capacité dans un filtre RC. Nous avons du prendre en compte l’influence de la sonde et l'oscilloscope. Les capacités sont plutôt proche de celle mesuré nous avons jusqu'à 20\% d'écart. Il existe des différences géométriques et électriques entre la simulation et la mesure qui explique les écarts. Ces résultats prouvent que la simulation est viable. Pour de futures simulations, nous saurons à quoi nous attendre si nous passons à la réalité. Une carte de développement du convertisseur choisi est ensuite utilisé. Il en ressort que la mesure du convertisseur est trop influencée par la capacité parasite à la masse. Une mesure absolue donne trop d'écart avec les résultats du filtre RC et la simulation. Nous avons pu cependant l'utilisé en relatif afin de tester l'effet de goutte d'eau. Le capteur les détectent différemment suivant leurs positions et leurs tailles. Un travail d'optimisations sera nécessaire sur le peigne.

L'objectif de ce travail était d'effectuer une pré-étude afin de valider le concept et de définir un premier cahier des charges. Ces deux points ont parfaitement été complété. Une analyse a été faite pour définir le cahier des charges et les différentes simulations et mesures nous prouvent que le capteur est totalement faisable. De plus, ce travail nous a apporté une compréhension des phénomènes en jeux, ainsi qu'une certaine confiance dans la simulation et les techniques de mesures. Ce sera un très bon départ pour la suite du développement du capteur.

Afin d'aboutir à un produit fini, une simulation plus poussée sera nécessaire pour optimiser le peigne et la détection de petite goutte. Des tests dans le plus de conditions possibles devrons être effectué afin de tester l’influence de l'humidité de l'air sur la mesure. Une réévaluation du choix du convertisseur devra être faite. Le circuit complet sera développé en faisant attention à la gestion de l'alimentation. Le programme du microcontrôleur pour l'acquisition et l'interface devra être écrit. Et enfin, des test finaux devrons être effectués afin de préparer le capteur à une certification pour une future commercialisation. Même si beaucoup de travail sera encore nécessaire pour mener à terme ce projet, cette pré-étude est très encourageante pour la suite. 
