\documentclass[../pa_rapport.tex]{subfiles}

\usepackage{textcomp, gensymb}
% Entête et pieds de page (A ADAPTER)
\usepackage{fancyhdr}
\setlength{\headheight}{27.06pt}
\pagestyle{fancy}
% Entête :
\renewcommand{\headrulewidth}{1pt}
\fancyhead[L]{\titre} 
\fancyhead[R]{} 
% Pied de page
\pagenumbering{arabic}
\fancyfoot{} % clear all footer fields
\renewcommand{\footrulewidth}{1pt}
\fancyfoot[R]{Page \thepage / \pageref*{LastPage}} 
\fancyfoot[L]{\nomAuteur }
% debut du corps du rapport 

\begin{document}

		
	
\begin{center}
%\begin{tabular}{>{\raggedright\arraybackslash} m{.4\textwidth}>{\raggedleft\arraybackslash} m{.5\textwidth}}

\begin{minipage}{\textwidth}
	\centering
  $\vcenter{\hbox{\includegraphics[width=8cm]{../images/mse-full-cropped.pdf}}}$
\hfill
  $\vcenter{\hbox{\includegraphics[height=2cm]{../images/logo_hesso.pdf}}}$
\end{minipage}

%\end{tabular}
\end{center}

\begin{small}
\begin{raggedright}
Master of Science HES-SO in Engineering\\
Av. de Provence 6\\
CH-1007 Lausanne
\end{raggedright}
\end{small}

\vfill

\raggedleft
\begin{tabular}{r}
\huge Master of Science HES-SO in Engineering\\
\\
\huge Orientation : Electrical Engineering (EIE)
\end{tabular}

\vfill

\begin{raggedleft}
	\Huge Génération d'un signal MR-OFDM avec une SDR (software defined radio) pour tester des récepteurs MR-OFDM
\end{raggedleft}
\vfill
\begin{raggedleft}
\large Fait par\\
\Huge Sébastien Deriaz\\[20pt]
\large Sous la direction de Pierre Favrat\\
Dans l'institut IICT de la HEIG-VD
\end{raggedleft}
\vfill

\begin{raggedleft}
\large Expert externe Didier Nicoulaz, Phd
\end{raggedleft}

\vfill
\large Lausanne, HES-SO Master, 2022

\thispagestyle{empty}
%	\setcounter{page}{1}
%\fancyfoot[R]{} 
%\setcounter{page}{2}
\newpage
\thispagestyle{empty}
\ 
\newpage
\thispagestyle{empty}
\raggedright
Accepté par la HES-SO Master (Suisse, Lausanne) sur proposition de\\[40pt]
Pierre Favrat, conseiller du projet d'approfondissement\\[40pt]

Lausanne, le \today\\[40pt]
\begin{center}
\begin{tabularx}{\textwidth}{
>{\raggedright\arraybackslash}X
c
>{\raggedright\arraybackslash}X
}
Pierre Favrat & \hspace{20pt} & Philippe Barrade\\
Conseiller du PA & \hspace{20pt} & Responsable de la filière Electrical Engineering\\
%\vspace*{10pt}
%\\
%\hrulefill &  & \hrulefill
\end{tabularx}
\end{center}


\pagebreak
\thispagestyle{empty}
\section*{Remerciements}
Je tiens à remercier mon professeur encadrant, M. Pierre Favrat qui m'a permis de développer mes compétences tout au long d'un travail enrichissant et passionnant.\\
Je souhaite également exprimer ma reconnaissance envers Yann Charbon, dont les conseils avisés m'ont grandement aidé lors de la mise en œuvre du travail.\\
Enfin, un grand merci également à ma famille et à mes amis pour leur soutien tout au long de ce travail.
\thispagestyle{empty}
\tableofcontents
\pagebreak
